\documentclass[handout]{beamer}
\usepackage{graphicx}
\usepackage[utf8]{inputenc}
\usepackage[russian]{babel}
\usepackage{xcolor}
\usepackage{amsmath,amssymb,amsthm}
\usetheme{Boadilla}

\colorlet{beamer@blendedblue}{green!40!black}

\setbeamertemplate{navigation symbols}{}
\title{Спецкурс 2020/2021: ``Геометрические и комбинаторные свойства матриц и
аппроксимация'' \\ Блок лекций ``Сложность матриц и аппроксимация'' \\ Лекция 1:
``Коммуникационная сложность''}
\renewcommand\le{\leqslant}
\renewcommand\ge{\geqslant}

\newcommand\R{\mathbb R}
\newcommand\N{\mathbb N}
\newcommand\Z{\mathbb Z}
\newcommand\T{\mathbb T}
\newcommand\CC{\mathbb C}
\newcommand\eps{\varepsilon}


\newcommand{\tvec}{\mathbf{t}}

\newcommand\E{\mathsf E}
\renewcommand\P{\mathsf P}

%\newtheorem*{theorem}{Теорема}
%\newtheorem{lemma}{Лемма}
\newtheorem*{statement}{Statement}
%\newtheorem*{conjecture}{Гипотеза}
%\newtheorem*{remark}{Замечание}

\DeclareMathOperator{\conv}{conv}
\DeclareMathOperator{\rank}{rank}
\DeclareMathOperator{\rig}{rig}
\DeclareMathOperator{\sign}{sign}
\DeclareMathOperator{\Var}{Var}
\DeclareMathOperator{\Arg}{Arg}
\DeclareMathOperator{\Real}{Re}
\renewcommand\C{\mathrm{C}}

\begin{document}
\maketitle

\begin{frame}{Коммуникационная сложность (Yao, 1979)}
    
    Пусть $\mathcal X$, $\mathcal Y$~--- два конечных множества,
    $f\colon\mathcal X\times\mathcal Y\to\{0,1\}$. Рассмотрим
    задачу коммуникации. Есть два участника, Анна и Борис, их задача: по
    выданным $x$ и $y$ вычислить $f(x,y)$. Трудность в том, что $x$ известно
    только Анне, а $y$~--- только Борису. Им разрешается обмениваться сообщениями:
    Анна посылает $a_1=A_1(x)$, Борис в ответ $b_1=B_1(y,a_1)$, и т.д. Правило
    составления сообщений~--- протокол~--- должно гаранировать восстановление
    $f(x,y)=b_t$ на некотором шаге.
    \pause

    Через $\C(P)$ обозначим суммарную длину (в битах) сообщений в худшем случае, если
    используется протокол $P$. Минимально возможное значение $\C(P)$ по всем
    протоколам называется \textit{коммуникационной сложностью} $f$ и обозначается через $\C(f)$:
    $$
    \C(f) := \min_P \max_{(x,y)\in\mathcal X\times\mathcal Y} \sum_{i\le t(P;x,y)}
    \mathrm{len}(a_i(P;x,y)) + \mathrm{len}(b_i(P;x,y)).
    $$
\end{frame}

\begin{frame}
    Формализации понятия протокола могут немного отличаться. Для нас это не
    играет роли, т.к. мы интересуемся величиной $\C$ с точностью до
    мультипликативной постоянной. Один из вариантов определения протокола: бинарное дерево, в каждой
    из вершин либо $A_v\colon\mathcal X\to\{0,1\}$, либо $B_v\colon\mathcal
    Y\to\{0,1\}$, в листе ответ.
    \pause\vspace{5pt}

    Тривиальная оценка:
    $$
    \C(f)\le \lceil \log_2 |\mathcal X|\rceil + 1.
    $$
    Действительно, Анна кодирует элемент $x\in\mathcal X$ с помощью 
    $\lceil \log_2 |\mathcal X|\rceil$ бит и отправляет Борису.
    Борис восстанавливает $x$ и отправляет Анне $f(x,y)$.

\end{frame}

\begin{frame}{Одноцветные прямоугольники}

    Получим оценку снизу на $\C(f)$. Введём понятие \textit{истории} сообщений:
    $h=(a_1,b_1,a_2,\ldots)$. Для оптимального протокола всего не более
    $2^{\C(f)}$ историй(*). Пусть $R_h=\{(x,y)\mapsto h\}$.
    \begin{itemize}
        \item $f$ на $R_h$ либо всюду $1$, либо $0$ (т.е. множество $R_h$
            ``одноцветно'');
        \item $R_h$ не пересекаются и их объединение есть $\mathcal
            X\times\mathcal Y$.
        \item $R_h$ есть прямоугольник, т.е. имеет вид $I\times J$.
    \end{itemize}
    \pause
    Поясним последнее свойство. Пара $(x,y)$ попадает в историю
    $h=(a_1,b_1,\ldots)$, когда
    выполнены условия:
    $$
    A_1(x) = a_1,\;B_1(y,a_1)=b_1,\;A_2(x,a_1,b_1)=a_2,\ldots
    $$
    Т.е. условия распадаются на зависящие только от $x$ (нечётные) и от $y$
    (чётные).
    \pause\vspace{5pt}

    Через $\chi(f)$ обозначим минимальное количество $f$-одноцветных
    прямоугольников, на которые можно разбить $\mathcal X\times\mathcal Y$.
    Из вышесказанного следует:

    $$
    \chi(f)\le 2^{\C(f)}.
    $$

\end{frame}

\begin{frame}{Пример: $\mathrm{EQ}$}

    Пусть $\mathcal X=\mathcal Y$ и $|X|=N$. Положим
    $\mathrm{EQ}_N(x,y)=1$, если $x=y$ и $0$ иначе. Утверждение:
    $\C(\mathrm{EQ}_N)\asymp \log N$.

    \pause\vspace{5pt}

    Оценка сверху тривиальна. Оценка снизу следует из $\C(f)\ge\log_2\chi(f)$.
    Рассмотрим разбиение на одноцветные прямоугольники. Прямоугольник с
    $\mathrm{EQ}_N=1$ может содержать только одну точку $(x,x)$, следовательно,
    их не меньше $N$ и $\chi(\mathrm{EQ}_N)\ge N$.

\end{frame}

\begin{frame}
    Имеет место обратное неравенство (Aho, Ullman, Yannakakis, 1983):
    $$
    \C(f)\ll \log^2\chi(f).
    $$
    \textbf{Упражнение.} Докажите неравенство. (Анна и Борис должны определить прямоугольник,
    в котором находятся.)
    \includegraphics[width=15cm]{rect.png}
\end{frame}

\begin{frame}{Связь с рангом}
    Занумеруем множества: $\mathcal X=\{x_1,\ldots,x_M\}$ и $\mathcal
    Y=\{y_1,\ldots,y_N\}$. Тогда функцию $f$ можно отождествить с матрицей из
    $\{0,1\}^{M\times N}$:
    $$
    f\quad \longleftrightarrow\quad (f(x_i,y_j))_{\substack{1\le i\le M\\1\le j\le N}}.
    $$
    Мы будем пользоваться матричной терминологией, не оговаривая этого особо.
    \pause\vspace{5pt}

    Утверждение: $\chi(f) \ge \rank f$. Действительно, матрица $f$ представляется
    в виде суммы матриц $f|_{R_h}$. Каждая из них имеет ранг 1 в силу того, что
    $R_h$~--- одноцветный прямоугольник. Следствие:
    $$
    \C(f) \ge \log_2\chi(f) \ge \log_2 \rank f.
    $$

    Пример. Для любой невырожденной $N\times N$ матрицы имеем $\C(f)\asymp\log
    N$.
\end{frame}

\begin{frame}{Нерешенные проблемы}

    Итак, $\log\chi(f) \ll \C(f) \ll (\log\chi(f))^2$. 

    \textbf{Проблема 1.} Верно ли, что $\C(f) \ll \log\chi(f)$?
    \pause\vspace{10pt}

    \textbf{Гипотеза.} Верно ли, что $\log\chi(f)\ll\log\rank f$?
    \pause\vspace{5pt}

    Гипотеза была опровергнута в серии публикаций (Alon-Seymour, Raz-Spieker, Razborov).
    \pause\vspace{5pt}

    \textbf{Проблема 2} (log-rank гипотеза). Верно ли, что $\log\chi(f)\ll (\log\rank f)^C$?
\end{frame}

\begin{frame}{Вероятностные модели}

    Предположим, Анна и Борис имеют доступ к последовательностям случайных бит
    (достаточно большой длины) и могут использовать эти биты во время исполнения
    протокола (или: подбрасывать монету). Тогда результат работы протокола $Q$ на входе $(x,y)$ это
    случайная величина $Q(x,y)$.
    
    Скажем, что протокол $Q$ вычисляет $f$ с ошибкой $\eps$, если
    $$
    \forall (x,y)\in\mathcal X\times\mathcal Y\quad \mathsf{P}(Q(x,y)\ne
    f(x,y))\le \eps.
    $$
    (Ясно, что можно обеспечить $\eps=1/2$, просто бросая монетку.)
    \pause\vspace{5pt}

    Минимальная сложность протокола, вычисляющего $f$ с ошибкой $\le 1/3$, 
    обозначается $R(f)$ и называется \textit{вероятностной коммуникационной
    сложностью $f$ в модели с ограниченной ошибкой}.

    Ошибку $\eps=1/3$ можно превратить в $\eps=10^{-100}$, повторив
    действия достаточное количество раз.

\end{frame}

\begin{frame}{Сложность вычисления $\mathrm{EQ}_N$ с ограниченной ошибкой}
    \begin{theorem}[Yao, Rabin]
        $R(\mathrm{EQ}_N)\ll \log\log N$.
    \end{theorem}

    Доказательство.
        Закодируем элементы $\mathcal X$ двоичными векторами $x\in\{0,1\}^n$,
        где $n=\lceil\log_2N\rceil$. Вектор $x$ отождествим с многочленом
        $x_1+x_2\xi+\ldots+x_n\xi^{n-1}$. Таким образом, у Анны и Бориса есть
        многочлены $g(\xi)$ и $h(\xi)$ и они хотят определить, равны ли эти
        многочлены.
        \pause\vspace{5pt}

        Заранее фиксируется простое число $p\in[3n,cn]$. Анна выбирает случайное
        $\xi\in\{0,1,\ldots,p-1\}$ и отправляет пару $(\xi,g(\xi)\bmod p)$ Борису.
        Борис вычисляет $h(\xi)\bmod p$ и выдаёт ``1'', если значения совпали и
        ``0'' в противном случае.

        По протоколу передаётся $\asymp\log p\asymp\log\log N$ бит, как и
        требовалось. Какова вероятность успеха?
\end{frame}


\begin{frame}{Сложность вычисления $\mathrm{EQ}_N$ (продолжение)}
    Если $g=h$, то выдаётся ``1'' безо всякой ошибки.

    Если $g\ne h$, то $g-h$ это ненулевой многочлен степени не выше $n$, он
    имеет не более $n$ корней в $\mathbb{F}_p$, то есть количество $\xi$, таких
    что $g(\xi)=h(\xi)$, не превосходит $n\le p/3$. Значит, вероятность
    ошибиться не больше $1/3$.

\end{frame}


\begin{frame}{Дискрепанс}
    В детерминированной модели мы рассматривали $f$-одноцветные прямоугольники
    (и доказывали, что их должно быть много, т.к. они в том или ином смысле малы). В вероятностной модели нужно
    рассматривать произвольные прямоугольники $R$. Пусть $N_0(f,R)$~---
    количество точек в $R$, для которых $f=0$, а $N_1(f,R)$~--- количество точек
    с $f=1$. Для нижних оценок сложности нам нужно доказать, что если
    прямоугольник большой, то он ``сбалансированный'', то есть $N_0$ и $N_1$
    близки друг к другу.
    \pause

    Дискрепансом функции $f\colon\mathcal X\times\mathcal Y\to\{0,1\}$ назовём
    величину
    $$
    \mathrm{disc}_u(f)=\max_R\frac{|N_0(f,R)-N_1(f,R)|}{|\mathcal
    X|\times|\mathcal Y|}.
    $$
    Эта величина измеряет отклонение $f$ от равномерного
    распределения, для семейства комбинаторных прямоугольников.
    \pause
    Возможно обобщение на другие распределения $\mu$ на множестве $\mathcal
    X\times\mathcal Y$:
    $$
    \mathrm{disc}_\mu(f)=\max_R|\mu(R\cap\{f=0\})-\mu(R\cap\{f=1\})|
    $$
    (нам встретится эта величина позже).
\end{frame}

\begin{frame}{Дискрепанс}
    $$
    \mathrm{disc}_u(f)=\max_R\frac{|N_0(f,R)-N_1(f,R)|}{|\mathcal
    X|\times|\mathcal Y|}.
    $$
    Если $R$~--- одноцветный прямоугольник, то $\mathrm{disc}_u(f)\ge
    |R|/|\mathcal X\times Y|$. Суммируя по одноцветным прямоугольникам, получим
    $\chi(f)\mathrm{disc}_u(f)\ge 1$, то есть
    $$
    \chi(f)\ge 1/\mathrm{disc}_u(f),\quad \C(f)\ge \log_2(1/\mathrm{disc}_u(f)).
    $$
    \pause\vspace{5pt}
    Оказывается, имеет место намного более сильный результат:
    $$
    R(f)\gg \log(1/\mathrm{disc}_u(f)).
    $$
\end{frame}

\begin{frame}
    Пример: $\mathrm{IP}_n(x,y)=\sum x_iy_i\bmod 2$, $x,y\in\{0,1\}^n$.
    \pause\vspace{5pt}

    Пусть $M$~--- матрица из $\{\pm1\}^{N\times N}$. Дискрепанс определяется
    аналогично, как максимум по комбинаторным прямоугольникам $R$ величины
    $N^{-2}|\sum_{(i,j)\in R} M_{i,j}|$.
    \pause
    Применим линейную алгебру! Пусть $R=A\times B$, тогда дискрепанс 
    равен
    \begin{multline*}
    N^{-2}|\sum_{(i,j)\in R} M_{i,j}| = N^{-2}|\mathsf{1}_A^t M \mathsf{1}_B|\le \\
    \le N^{-2}\|\mathsf{1}_A\|_2\cdot\|\mathsf{1}_B\|_2\cdot\|M\|_{2\to 2} \le
    N^{-2}|A|^{1/2}|B|^{1/2}\|M\|_{2\to 2}.
    \end{multline*}
    Следовательно, $\mathrm{disc}_u(M) \le N^{-1}\|M\|_{2\to 2}$.
    \pause\vspace{5pt}

    Рассмотрим матрицу $W_n$, соответствующую $2\cdot\mathrm{IP}_n-1$ (т.е.
    значения $\{0,1\}$ заменили на $\{-1,1\}$). Получится так называемая матрица
    \textit{Уолша--Адамара}. Нетрудно видеть, что строки
    матрицы ортогональны и их длина равна $2^{n/2}$. Следовательно, $\|W_n\|_{2\to
    2} = 2^{n/2}$. Значит, $\mathrm{disc}_u(\mathrm{IP}_n)=\mathrm{disc}_u(W_n)\le 2^{-n/2}$, откуда
    $R(\mathrm{IP}_n)\gg n$.

    В отличие от $\mathrm{EQ}_N$, функция $\mathrm{IP}_n$ осталась сложной даже
    с разрешенной ограниченной ошибкой.
\end{frame}

\begin{frame}{Таблица результатов}
    $U(f)$~--- сложность с ``неограниченной ошибкой'' (вероятность успеха
    $>1/2$).
    \vspace{5pt}

    \begin{tabular}{|l|l|l|l|}
        \hline
        $f$ & $\C(f)$ & $R(f)$ & $U(f)$ \\
        \hline
        $\mathrm{EQ}$ & $\asymp\log N$ & $O(\log\log N)$ & $O(1)$ \\
        $\mathrm{DISJ}$ & $\asymp\log N$ & $\asymp\log N$ & $O(\log\log N)$ \\
        $\mathrm{IP}$ & $\asymp\log N$ & $\asymp\log N$ & $\asymp\log N$ (Forster) \\
        \hline
    \end{tabular}
\end{frame}


\begin{frame}{Список литературы}
    \begin{itemize}
        \item А.А.~Разборов, ``Коммуникационная сложность''.
        \item E.~Kushilevitz, N.~Nisan, ``Communication complexity''.
        \item S.~Arora, B.~Barak, ``Computational Complexity: A Modern Approach''.
    \end{itemize}
\end{frame}

\end{document}
