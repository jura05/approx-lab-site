\documentclass[handout]{beamer}
\usepackage[utf8]{inputenc}
\usepackage[russian]{babel}
\usepackage{xcolor}
\usepackage{amsmath,amssymb,amsthm}
\usetheme{Boadilla}

\colorlet{beamer@blendedblue}{green!40!black}

\setbeamertemplate{navigation symbols}{}
\title{Спецкурс 2020/2021: ``Геометрические и комбинаторные свойства матриц и
аппроксимация'' \\ Блок лекций ``Сложность матриц и аппроксимация'' \\ Лекция 7:
``Ранг тензоров''}
\renewcommand\le{\leqslant}
\renewcommand\ge{\geqslant}

\newcommand\R{\mathbb R}
\newcommand\N{\mathbb N}
\newcommand\Z{\mathbb Z}
\newcommand\T{\mathbb T}
\newcommand\CC{\mathbb C}
\newcommand\eps{\varepsilon}


\newcommand{\tvec}{\mathbf{t}}

\newcommand\E{\mathsf E}
\renewcommand\P{\mathsf P}

%\newtheorem*{theorem}{Теорема}
%\newtheorem{lemma}{Лемма}
\newtheorem*{statement}{Утверждение}
%\newtheorem*{conjecture}{Гипотеза}
%\newtheorem*{remark}{Замечание}

\DeclareMathOperator{\conv}{conv}
\DeclareMathOperator{\rank}{rank}
\DeclareMathOperator{\tr}{tr}
\DeclareMathOperator{\Rig}{Rig}
\DeclareMathOperator{\sign}{sign}
\DeclareMathOperator{\disc}{disc}
\DeclareMathOperator{\Var}{Var}
\DeclareMathOperator{\Arg}{Arg}
\DeclareMathOperator{\Real}{Re}
\DeclareMathOperator{\margin}{margin}
\DeclareMathOperator{\mc}{mc}

\begin{document}
\maketitle

\begin{frame}{Тензоры}
    Тензор порядка $d$ размера $n_1\times n_2\times\cdots\times n_d$~--- это массив
    чисел
    $$
    T = (T_{i_1,\ldots,i_d}),\quad i_1\in[n_1],\ldots,i_d\in[n_d].
    $$
    Напомним, $[n]:=\{1,2,\ldots,n\}$.
    \pause\vspace{5pt}

    Числа $n_1,\ldots,n_d$ называются \textit{размерностями} тензора. 
    Обозначаем множество таких тензоров как $\R^{n_1\times
    n_2\times\ldots \times n_d}$.
    \pause

    Очевидно, тензор размерности $(n_1,\ldots,n_d)$ состоит из $N=n_1n_2\ldots
    n_d$ элементов. 
    \pause\vspace{5pt}

    Можно считать, что тензор~--- это функция нескольких (дискретных)
    аргументов.
    \pause\vspace{5pt}

    \begin{itemize}
        \item скаляр $x\in\R$~--- тензор порядка $0$;\pause
        \item вектор $x\in\R^n$~--- тензор порядка $1$;\pause
        \item матрица $x\in\R^{n_1\times n_2}$~--- тензор порядка $2$;\pause
        \item тензор $x\in\R^{n_1\times\ldots\times n_d}$ порядка $d$.
    \end{itemize}

\end{frame}
\begin{frame}{Алгебраический взгляд}

    Матрицу можно отождествить с линейным оператором, это даёт много полезных и
    важных понятий (собственные и сингулярные числа, операторные нормы, инвариантность следа и
    определителя и т.п.). Каков алгебраический взгляд на тензоры?
    \pause\vspace{5pt}

    В более общем смысле тензор~--- это элемент тензорного произведения
    пространств. Тензорное произведение линейных пространств $U$ и $V$ состоит
    из (формальных) линейных комбинаций выражений вида $u\otimes v$, $u\in U$, $v\in V$, в
    котором некоторые комбинации отождествлены~--- так, чтобы выполнялись
    условия билинейности
    $$
    (\lambda_1 u_1+\lambda_2 u_2)\otimes v = \lambda_1 (u_1\otimes v) +
    \lambda_2 (u_2 \otimes v),
    $$
    и аналогично для второго аргумента.
    \pause

    Тензоры вида $u\otimes v$ называются \textit{разложимыми}, они порождают всё
    тензорное произведение $U\otimes V$ (но составляют лишь малую его часть).
    \pause

    Можно показать, что если $\{u_1,\ldots,u_m\}$~--- базис $U$, а
    $\{v_1,\ldots,v_m\}$~--- базис $V$, то $\{u_i\otimes v_j\}$ будет базисом
    $U\otimes V$. Значит, $\dim(U\otimes V)=\dim(U)\dim(V)$. Аналогично
    для нескольких пр-в.
\end{frame}

\begin{frame}{Тензорное произведение и функции многих переменных}
    Выберем в пространстве $U$ базис $\{u_i\colon i\in I\}$. Тогда $U$ можно
    отождествить с функциями $f\colon I\to\R$. Аналогично для
    $V$ с базисом $\{v_j\colon j\in J\}$: $g\colon J\to\R$.
    \pause\vspace{5pt}

    В тензорном произведении $U\otimes V$ базис~--- $\{u_i\otimes v_j\}$, поэтому
    $U\otimes V$ можно отождествить с пространством функций $h\colon I\times J\to \R$.
    Разложимые тензоры соответствуют функциям вида $h(i,j)=f(i)g(j)$.
    \pause\vspace{5pt}

    То есть, тензорное произведение для пространств функций соответствует
    переходу к декартовому произведению областей определения.
    \pause\vspace{5pt}

    \begin{itemize}
        \item многочлены: $\R[x]\otimes\R[y]\cong \R[x,y]$;\pause
        \item разложимые функции от $d$ переменных:
        $f_1(x_1)f_2(x_2)\ldots f_d(x_d)$;\pause
        \item многомерная тригонометрическая система $\exp(i(k_1x_1+\ldots
            k_dx_d))$ состоит из разложимых функций;\pause
    \item для функциональных пространств то же самое, например,
        $C(X)\otimes C(Y)=C(X\times Y)$ для хороших (компактных) $X$, $Y$ и при
            подходящем определении нормы тензорного произведения.
    \end{itemize}

\end{frame}

\begin{frame}{Алгебраический взгляд}
    В линейной алгебре тензоры (в нашем понимании) возникают как координаты
    элементов тензорных произведений. Пусть $U_1,\ldots,U_d$~--- линейные
    пространства с базисами $\{u^1_1,\ldots,u^1_{n_1}\}$, \ldots,
    $\{u^d_1,\ldots,u^d_{n_d}\}$, соответственно. Тогда вектора
    $$
    u^1_{i_1}\otimes u^2_{i_2}\otimes \cdots \otimes u^d_{n_d}
    $$
    образуют базис пространства $U_1\otimes U_2\otimes\cdots\otimes U_d$.
    \pause

    Значит, $\mathcal T\in U_1\otimes\cdots\otimes U_d$ раскладывается по базису:
    $$
    \mathcal T = \sum_{i_1\in[n_1],\ldots i_d\in[n_d]}
    T_{i_1,\ldots,i_d}u^1_{i_1}\otimes\cdots\otimes u^d_{n_d}.
    $$
    \pause
    При замене базисов в $U_j$ координаты тензора преобразуются по известным
    законам; при этом некоторые свойства инвариантны относительно замены базиса
    (см. далее).
    \pause

    Пространство $\R^n$ обладает естественным базисом; мы можем
    отождествить тензорное произведение (в алгебраическом смысле)
    с пространством тензоров (как массивов чисел):
    $
    \R^{n_1}\otimes\ldots\otimes\R^{n_d} \cong \R^{n_1\times\ldots\times n_d}.
    $

\end{frame}

\begin{frame}{Пример: оператор}
    Тензорное произведение $V^*\otimes V$ можно отождествить с пространством
    операторов на $V$:
    $$
    V^*\otimes V \cong L(V,V),\quad (\xi\otimes v)\colon u\mapsto \xi(u)v.
    $$
    \pause
    Выберем в $V$ базис $\{e_j\}$, в $V^*$ возникает сопряжённый базис
    $\{\xi_i\}$, т.е. такой что $\xi_i(e_j)=\delta_{i,j}$. Тензор раскладыватеся
    по базису:
    $$
    V^*\otimes V \ni \mathcal T = \sum_{i,j} T_{i,j}\xi_i\otimes e_j.
    $$
    \pause
    Вычислим $\mathcal Te_k$:
    $$
    \mathcal Te_k = (\sum T_{i,j}\xi_i\otimes e_j)e_k = \sum_{i,j}
    T_{i,j}\xi_i(e_k)e_j = \sum_j T_{k,j}e_j.
    $$
    \pause
    Таким образом, $(T_{i,j})$ это транспонированная матрица оператора $\mathcal T$.
\end{frame}


\begin{frame}{Пример: билинейное отображение}
    Пространство $L(U\times V,W)$ билинейных отображений $\mathcal T\colon U\times V\to
    W$ отождествляется с тензорным произведением $U^*\otimes V^* \otimes W$.
    Выбор базисов $\{u_i\}$, $\{v_j\}$, $\{w_k\}$ даёт разложение
    $$
    \mathcal T = \sum_{i,j,k}T_{i,j,k}\xi_i\otimes \eta_j\otimes w_k.
    $$
    \pause\vspace{5pt}

    При этом 
    $$
    \mathcal T(u_p,v_q) = \sum_{i,j,k}T_{i,j,k}\xi_i(u_p)\eta_j(v_q)w_k = \sum_k
    T_{p,q,k}w_k.
    $$
    \pause\vspace{5pt}

    К сожалению, для тензоров порядка $d\ge3$ нет настолько же удобного
    соответствия, как соответствие между матрицами и операторами.
\end{frame}

\begin{frame}{Ранг матрицы}
    Напомним эквивалентные определения ранга матрицы:
    \begin{itemize}
        \item размерность образа $\dim\mathrm{Im}\mathcal A$ оператора с
            матрицей $A$;\pause
        \item размерность пространства строк/столбцов;\pause
        \item максимальный размер невырожденного минора:
            $\max\{|I|=|J|\colon \det A[I,J]\ne 0\}$;
        \item минимальная размерность $r$, в которой найдутся вектора
            $x_i\in\R^r$, $y_j\in\R^r$, такие что $A_{i,j}=\langle
            x_i,y_j\rangle$;
        \item минимальное число одноранговых матриц (т.е. вида $R_{i,j}=a_ib_j$)
            в представлении $A=R^{(1)}+R^{(2)}+\ldots+R^{(r)}$.
    \end{itemize}
    \pause\vspace{5pt}
    Обобщим понятие ранга на основании последнего определения.
\end{frame}

\begin{frame}{Ранг тензора}
    Назовём тензор $T\in\R^{n_1\times\cdots\times n_d}$ \textit{разложимым}, если
    $$
    T_{i_1,\ldots,i_d}=v^1_{i_1}v^2_{i_2}\ldots v^d_{i_d},\quad
    i_1\in[n_1],\ldots,i_d\in[n_d],
    $$
    для некоторых векторов $v^1,\ldots,v^d$. Например, для тензора порядка 3:
    $$
    T_{i,j,k} = a_ib_jc_k,\quad a\in\R^{n_1},b\in\R^{n_2},c\in\R^{n_3}.
    $$
    \pause\vspace{5pt}

    Рангом тензора назовём минимальное количество разложимых тензоров в
    представлении $T=R^{(1)}+R^{(2)}+\ldots+R^{(r)}$. Таким образом,
    ``разложимый'' это синоним слову ``одноранговый'' (кроме нулевого).
    \pause\vspace{5pt}

    Пример: пусть $n_1=n_2=n_3$. \textcolor{blue}{Чему равен ранг тензора
    $T_{i,j,k}=\mathbf{1}\{i=j=k\}$?}\pause~Ответ: $\rank T=n$.
\end{frame}

\begin{frame}{Ранг в общем виде}
    Сформулируем понятие ранга в общем, инвариантном виде. Пусть
    $$
    \mathcal T\in U_1\otimes U_2\otimes\cdots\otimes U_d.
    $$
    Напомним, что разложимыми тензорами называются тензоры вида $u_1\otimes
    u_2\otimes\cdots\otimes u_d$, $u_j\in U_j$. Они порождают всё тензорное
    произведение (но составляют малую его часть).
    \pause\vspace{5pt}

    Ранг тензора $\mathcal T$ в алгебраическом смысле равен рангу его координат
    $T\in\R^{n_1\times\cdots\times n_d}$ в фиксированном базисе. При этом
    понятия разложимости и ранга не зависят от выбора базиса!
\end{frame}


\begin{frame}{Свойства ранга}
    Известно, что множество матриц ранга не выше $r$ замкнуто. Действительно,
    если $\rank A>r$, то некоторый $(r+1)\times (r+1)$ минор невырожден, это же
    верно и в окрестности матрицы.
    \pause\vspace{5pt}

    В пространстве тензоров естественным образом возникает топология. Можно
    рассматривать различные нормы, например, норму Фробениуса
    $$
    \|T\|_F^2 := \sum_{i_1,\ldots,i_d} T_{i_1,\ldots,i_d}^2.
    $$
    \pause\vspace{5pt}

    Множество разложимых тензоров замкнуто: если, скажем,
    $T^{(n)}=a^{(n)}\otimes b^{(n)}\otimes c^{(n)}$ и $T^{(n)}\to T$, то,
    нормируя $\|b^{(n)}\|_F=\|c^{(n)}\|_F = 1$, мы получим, что и нормы
    $\|a^{(n)}\|$ ограничены.\pause~Поэтому можно выбрать сходящуюся
    подпоследовательность
    $$
    a^{(n_k)}\to a,\quad b^{(n_k)}\to b,\quad c^{(n_k)}\to c,
    $$
    откуда $T^{(n)}\to a\otimes b\otimes c$.
    \pause

    В общем случае замкнутости уже нет.
\end{frame}

\begin{frame}
    \begin{statement}
        В пространстве $\R^{n_1\times n_2\times n_3}$ при $n_j\ge3$ существует
        последовательность тензоров ранга $2$, предел которой равен тензору
        ранга $3$.
    \end{statement}
    В~(\ref{zz}) слева стоят тензоры ранга 2, а справа~--- ранга 3 (при правильном
    выборе $x_i$, $y_i$):
    \begin{multline}
        \label{zz}
    \lim_{\eps\to0}\frac1\eps((x_1+\eps y_1)\otimes(x_2+\eps
    y_2)\otimes(x_3+\eps y_3) - x_1\otimes x_2\otimes x_3) =\\
    x_1\otimes x_2\otimes y_3 + x_1\otimes y_2\otimes x_3 + y_1\otimes
        x_2\otimes x_3. 
    \end{multline}
    \pause

    Проблема состоит в том, что одноранговые слагаемые имеют большую норму, но
    непостижимым образом ``сокращаются'' друг с другом и норма суммы оказывается
    малой.\pause~Если рассматривать \textit{регулярные} представления,
    в которых нормы слагаемых ограничены:
    $$
    T = \sum_{s=1}^r a^{(s)}_1\otimes a^{(s)}_2\otimes\cdots\otimes
    a^{(s)}_d,\quad
    \max_s \|a^{(s)}_1\|_F\cdot\|a^{(s)}_2\|_F\cdots\|a^{(s)}_d\|_F < C,
    $$
    то подобного эффекта не будет.
\end{frame}


\begin{frame}{Граничный ранг (border rank)}

    Скажем, что тензор $T$ имеет \textit{граничный ранг} не выше $r$ , если $T$
    является пределом некоторой последовательности тензоров ранга не выше $r$.
    Обозначение:
    $\underline{\rank}(T)$.
    \pause\vspace{5pt}

    Например, для $T=x\otimes y+x\otimes z+y\otimes z$ имеем $\rank(T)=3$,
    $\underline{\rank}(T)=2$.
    \pause\vspace{5pt}

    \textbf{Упражнение}. Придумайте тензор $T$ с $\rank T-\underline{\rank}T\ge
    2$.
    \pause\vspace{5pt}

    Известно (2017?), что отношение $\rank(T)/\underline{\rank T}$ может быть сколь
    угодно большим.
\end{frame}


\begin{frame}{Тензор матричного умножения}
    Рассмотрим
    операцию умножения матриц подходящего размера. Мы ``вытянем'' матрицы в
    длинные вектора, т.е. будем считать индексы $A_{i,j}$ одним индексом
    $(i,j)$.
    \pause\vspace{5pt}
    Тензор $M$ умножения матриц имеет вид
    $$
    C=AB \quad\Leftrightarrow\quad C_{(i,j)} = \sum_{(k,l),(p,q)}
    M_{(i,j),(k,l),(p,q)}A_{(k,l)}B_{(p,q)}.
    $$
    Его размерности это $(n_1n_3,n_1n_2,n_2n_3)$.
    \pause

    Как мы знаем, тензор умножения равняется
    $$
    M_{(i,j),(k,l),(p,q)} = \mathbf{1}\{k=i,l=p,q=j\},
    $$
    тогда получается привычная формула $C_{i,j}=\sum A_{i,p}B_{p,j}$.
\end{frame}


\begin{frame}{Тензор умножения матриц}
    Предположим, ранг этого тензора равен $r$ и мы имеем разложение
    $$
    M = \sum_{s=1}^r \alpha^{(s)}\otimes \beta^{(s)}\otimes \gamma^{(s)}.
    $$
    \pause\vspace{5pt}
    Тогда
    \begin{multline*}
    C_{(i,j)} = \sum_{(k,l),(p,q)}\sum_{s=1}^r
        \alpha^{(s)}_{(i,j)}\beta^{(s)}_{(k,l)}\gamma^{(s)}_{(p,q)}
        A_{(k,l)}B_{(p,q)} = \\
        = \sum_{s=1}^r \alpha^{(s)}_{(i,j)}
        (\sum_{(k,l)}\beta^{(s)}_{(k,l)}A_{(k,l)})(\sum_{(p,q)}\gamma_{(p,q)}^{(s)}B_{(p,q)}).
    \end{multline*}
    \pause\vspace{5pt}

    Заметим, что множители с $\beta$ и $\gamma$ не зависят от $(i,j)$.
    Следовательно, достаточно вычислить эти числа, сделать $r$ умножений и найти
    все $C_{(i,j)}$ как линейные комбинации.
\end{frame}
\begin{frame}{Тензор умножения матриц $(2\times 2)$}
    Рассмотрим теперь матрицы $2\times 2$. Оказывается, ранг тензора умножения
    равен $7$! Это позволяет умножить две $2\times 2$ матрицы, сделав только $7$
    умножений (обычный алгоритм требует $8$ умножений). 
    \pause
    
    Но главное в том, что можно считать элементы матриц $2\times 2$ не числами,
    а матрицами (формула для умножения справедлива и для блоков)!\pause~Поэтому
    умножение матриц $(2n)\times(2n)$ сводится к $7$ умножениям матриц $n\times
    n$! Общая сложность алгоритма не $n^3$ (как у
    тривиального), а $n^{\log_2 7} = n^{2.807\ldots}$.
    Этот алгоритм придумал Volker Strassen (1969).
    \pause\vspace{5pt}
    
    Показатель степени улучшался за счёт рассмотрения тензоров умножения матриц
    большего размера.
    Есть также оценки, основанные на $\underline\rank$, а не на обычном ранге!
\end{frame}

\begin{frame}{Максимальный ранг}
    Каков максимально возможный ранг тензора из $\R^{n_1\times \ldots \times n_d}$?
    \pause
    \begin{statement}
        $$
        \frac{n_1n_2\cdots n_d}{n_1+\ldots+n_d} \le \max\rank T \le n_1n_2\cdots n_d/\max\{n_j\}.
        $$
    \end{statement}
    \pause
    Докажем оценку сверху. Пусть размерность $n_1$ максимальна. Мы можем представить $T$ в
    виде суммы
    $$
    T_{i_1,\ldots,i_d} =
    \sum_{i_2',\ldots,i_d'}T_{i_1,i_2',\ldots,i_d'}\delta_{i_2,i_2'}\cdots\delta_{i_d,i_d'}.
    $$
    Каждое слагаемое представляет собой разложимый тензор.
    \pause

    Оценка снизу получается из соображений размерности. Тензор ранга $r$
    задаётся $r$ наборами векторов
    $u^{(s)}_1\in\R^{n_1},\ldots,u^{(s)}_d\in\R^{n_d}$. Следовательно, такой
    тензор описывается $r(n_1+\ldots+n_d)$ параметрами.\pause~Чтобы покрыть всё
    $n_1n_2\ldots n_d$-мерное пространство $\R^{n_1\times\cdots\times n_d}$,
    должно быть выполнено неравенство
    $$
    r(n_1+\ldots+n_d) \ge n_1\cdots n_d.
    $$
\end{frame}

\begin{frame}{Максимальный ранг}
    В случае тензоров $\R^{n\times n\times \cdots\times n}$ порядка $d$
    максимальный ранг заключён между $n^{d-1}/d$ и $n^{d-1}$.
    \pause\vspace{5pt}

    \textbf{Проблема}. Придумать конструктивный пример тензора из $\R^{n\times
    n\times n}$ ранга $\gg n^{1+\eps}$, $\eps>0$.

    Strassen доказал, что сложность вычисления трилинейной формы, задаваемой
    тензором $T$, не меньше $C\rank T$. Поэтому суперлинейные оценки ранга дают
    интересные следствия в теории сложности вычислений. (Ср. с теоремой
    Valiant-а).
    \pause

    Известны явные конструкции тензоров лишь ранга $\ge 3n-o(n)$.
    \pause\vspace{5pt}

    Тензорный ранг, как мы видим, намного сложнее матричного. Доказано, что
    задача вычисления тензорного ранга при $d\ge 3$ надо полем $\mathbb
    Q$ является NP-трудной.
\end{frame}

\begin{frame}{Multiparty communication}
    Рассмотрим задачу коммуникации с $k\ge 2$ участниками.
    Каждому выдаётся элемент $x_i\in X_i$, $i=1,\ldots,k$. Требуется вычислить
    $f(x_1,x_2,\ldots,x_k)$.
    \pause

    Функцию
    $f\colon X_1\times\cdots\times X_k\to \{0,1\}$ можно отождествить с тензором
    из $\{0,1\}^{n_1\times n_2\times\cdots\times n_d}$, $N_j = |X_j|$.
    \pause

    Рассмотрим прямое обобщение случая $k=2$: участникам известны только свои
    входные данные. Будем считать, что они обмениваются информацией, записывая
    сообщения ``на доске'', так, чтобы все видели (broadcast).
    \pause
    
    Зафиксируем протокол. Как и раньше, введём понятие \textit{истории}
    сообщений. Множество входов, приводящих к данной истории:
    $$
    R_h = \{(x_1,\ldots,x_k)\colon
    P(x_1)=h_1,\;P(h_1,x_2)=h_2,\;P(h_1,h_2,x_3)=h_3,\ldots\}
    $$
    представляет собой обобщённый ``прямоугольник'': $R_h = I_1\times
    I_2\times\cdots\times I_d$. 
    \pause

    Отсюда, как и раньше, получаем оценку сложности
    $$
    D(f)\ge\log_2\rank f.
    $$
\end{frame}

\begin{frame}{Multiparty communication}
    Перейдём к модели коммуникации ``число на лбу''. Теперь каждый участник
    видит все $x_j$, кроме своего. Здесь у участников больше информации и
    появляются новые эффекты.
    \pause
    
    Пример. Функция равенства:
    $\mathrm{EQ}(x_1,\ldots,x_k)=\mathbf{1}\{x_1=x_2=\cdots=x_k\}$, для
    одинаковых множеств $X_1=\ldots=X_k$ мощности $N$.
    \pause

    В случае коммуникации двух участников мы доказали оценку сложности
    $D(\mathrm{EQ})\ge\log N$. \textcolor{blue}{Что будет в случае $k\ge 3$
    участников?}\pause~Сложность падает до $D(\mathrm{EQ})=2$. Следовательно,
    $\log\rank$ оценки уже нет.
    \pause\vspace{5pt}

    Однако, по-прежнему справедлива оценка через дискрепанс:
    $$
    D(f) \gg \log(1/\mathrm{disc}\,(f)).
    $$
    Вместо обобщённых прямоугольников, правда, приходится рассматривать более
    хитрые множества. Для $k=3$ это множества вида
    $$
    (I_1\times I_2\times X_3)\cap (I_1'\times X_2\times I_3')\cap (X_1\times
    I_2''\times I_3'').
    $$

\end{frame}

\begin{frame}
    \begin{thebibliography}{XXXX}
        \bibitem{H}
            W.~Hackbusch, \textit{Tensor Spaces and Numerical Tensor Calculus}.
        \bibitem{K}
            D.~Knuth, \textit{The Art of Computer Programming}, vol.2, \S4.6.
    \end{thebibliography}
\end{frame}

\end{document}
