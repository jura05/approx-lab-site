\documentclass[handout]{beamer}
\usepackage[utf8]{inputenc}
\usepackage[russian]{babel}
\usepackage{xcolor}
\usepackage{amsmath,amssymb,amsthm}
\usetheme{Boadilla}

\colorlet{beamer@blendedblue}{green!40!black}

\setbeamertemplate{navigation symbols}{}
\title{Спецкурс 2020/2021: ``Геометрические и комбинаторные свойства матриц и
аппроксимация'' \\ Блок лекций ``Сложность матриц и аппроксимация'' \\ Лекция 3:
``Жёсткие матрицы''}
\renewcommand\le{\leqslant}
\renewcommand\ge{\geqslant}

\newcommand\R{\mathbb R}
\newcommand\N{\mathbb N}
\newcommand\Z{\mathbb Z}
\newcommand\T{\mathbb T}
\newcommand\CC{\mathbb C}
\newcommand\eps{\varepsilon}


\newcommand{\tvec}{\mathbf{t}}

\newcommand\E{\mathsf E}
\renewcommand\P{\mathsf P}

%\newtheorem*{theorem}{Теорема}
%\newtheorem{lemma}{Лемма}
\newtheorem*{statement}{Statement}
%\newtheorem*{conjecture}{Гипотеза}
%\newtheorem*{remark}{Замечание}

\DeclareMathOperator{\conv}{conv}
\DeclareMathOperator{\rank}{rank}
\DeclareMathOperator{\Rig}{Rig}
\DeclareMathOperator{\sign}{sign}
\DeclareMathOperator{\Var}{Var}
\DeclareMathOperator{\Arg}{Arg}
\DeclareMathOperator{\Real}{Re}

\begin{document}
\maketitle

\begin{frame}{Жесткость матрицы и линейные схемы}

Жесткость матрицы (Rigidity):
$$
    \Rig(A,r) := \min_{\rank B\le r} \#\{(i,j)\colon A_{i,j}\ne B_{i,j}\}.
$$
\pause
    \textcolor{red}{Разминка!}
    \begin{itemize}
        \item вычислите $\Rig(\mathrm{Id}_n,r)$;
            \pause
        \item оцените $\Rig(A,r)$ сверху для произвольной матрицы $A\in\R^{n\times n}$;
            \pause
        \item оценить $\Rig(\Delta_n,n/100)$ для верхнетреугольной
            $\{0,1\}$-матрицы $\Delta_n$.
    \end{itemize}
            \pause
\end{frame}

\begin{frame}
    Понятие жёсткости возникло в теории сложности в контексте \textit{линейных
    схем}, вычисляющих
линейные функций $x\mapsto Ax$. Работа Leslie Valiant-а 1977 года.
\pause
    
Схема состоит из узлов, на входе $n$ узлов-переменных
$x_1,\ldots,x_n$, на выходе должны быть узлы $y_i$, так чтобы
$(y_1,\ldots,y_n)=Ax$, промежуточные узлы: элементы сложения (с двумя
    входами) и умножения на скаляр. Узлы соединены в \textbf{ориентированный
    ацикличный граф}.
\pause
    \begin{itemize}
        \item размер схемы = количество рёбер;
        \item глубина = длина максимального пути;
    \end{itemize}

Можно вычислить любое отображение $\mathbb{F}^n\to\mathbb{F}^n$ схемой размера
$O(n^2)$ и глубины $O(\log n)$. \textcolor{red}{Почему?}

\end{frame}
\begin{frame}

\begin{theorem}[Valiant]
    Если линейное отображение $A\colon\mathbb{F}^n\to\mathbb{F}^n$ можно вычислить схемой размера $s$ и
    глубины $d$, то для любого $t>1$
    $$
    \Rig(A,\frac{s\log t}{\log d}) \le n2^{Cd/t}.
    $$
\end{theorem}
\pause

    \textbf{Следствие}: если для некоторых $\eps,\delta>0$ имеем $\Rig(A,\eps
    n)\ge n^{1+\delta})$, то схема, вычисляющая преобразование $x\mapsto Ax$ и имеющая
    логарифмическую глубину, имеет размер не менее $c(\eps,\delta)n\log\log n$.
    \pause

    Для случайных матриц над бесконечным полем $\Rig(A,r)=(n-r)^2$.

    \textbf{Проблема}: построить явное семейство жёстких матриц $n\times n$:
    $\Rig(A_n,\eps n)\ge n^{1+\delta}$.
    
\end{frame}


\begin{frame}
    Поработаем с ориентированными ацикличными графами.
    \pause\vspace{5pt}
    
    Назовём \textit{разметкой} графа $G=(V,E)$ отображение $L\colon V\to\mathbb
    Z$, такое что для всякого ребра $(u, v)\in E$ имеем $L(u) < L(v)$.
    \pause\vspace{5pt}

    Если есть разметка $L\colon V\to\{1,2,\ldots,d\}$, то глубина графа не
    больше $d$.
    \pause\vspace{5pt}

    Всякий граф глубины $d$ может быть размечен числами $\{1,2,\ldots,d\}$.
    \textcolor{red}{Почему?}
\end{frame}


\begin{frame}
    Для доказательства теоремы Valiant-а нам потребуется утверждение из теории
    графов. Его доказал...\pause Valiant.\pause

    \begin{lemma}
        Пусть $(V,E)$~--- ориентированный ацикличный граф глубины не выше $d$ и
        задано число $r$.
        Тогда можно удалить из графа не более
        $|E|\cdot r/\log_2 d$
        рёбер так, что глубина оставшегося графа будет не выше $d/2^r$.
    \end{lemma}
    В лемме для простоты считаем, что $d$ равно степени двойки.
\end{frame}
    
    \begin{frame}
    \begin{proof}
        Рассмотрим разметку $L\colon V\to\{0,1,\ldots,d-1\}$ и отождествим
        $\{0,1,\ldots,d-1\}$ с двоичными строками длины $\log_2d$ (двоичная
        запись числа).
        \pause\vspace{5pt}

        Возьмём ребро $(u,v)\in E$, тогда $L(u)<L(v)$.
        Пусть старший бит, где отличаются $L(u)$ и $L(v)$ это $i$-й бит.
        Через $E_i$ обозначим множество таких рёбер.
        \pause\vspace{5pt}

        Пусть мы удалили $E_i$. Тогда из разметки можно удалить $i$-й бит и
        свойство разметки будет выполнено!
        \pause\vspace{5pt}
        
        Т.к. разметка принимает $(\log_2d-1)$-битные значения, получится глубина
        не больше $d/2$.
        \pause\vspace{5pt}

        При удалении $r$ множеств $E_i$ получим глубину не более $d/2^r$.
        \pause\vspace{5pt}

        Выберем $E_{i_1},\ldots,E_{i_r}$ минимальной мощности. Выбираем $r$
        штук из $\log_2d$ с суммарной мощностью $|E|$, получим
        $|E_{i_1}|+\ldots+|E_{i_s}|\le |E|\cdot r/\log_2 d$.
    
    \end{proof}

\end{frame}

\begin{frame}
    Перейдём к доказательству теоремы Valiant-а. Пусть для матрицы $A$ есть
    схема размера $s$; приблизим $A$ матрицей малого ранга.
    \pause\vspace{5pt}

    Положим $t=2^r$ и удалим $m\le|E|\cdot r/\log_2d=sr/\log_2d$ рёбер так, чтобы
    остался граф глубины не более $d/t$.
    \pause\vspace{5pt}

    В каждой вершине графа вычисляется линейная форма от входов
    $x_1,\ldots,x_n$. Пусть $b_1,\ldots,b_{m'}$~--- линейные формы в вершинах на
    концах удалённых рёбер ($m'\le m$).
    \pause\vspace{5pt}

    Рассмотрим фиксированную выходную вершину. Как она вычисляется? Пройдём от неё вверх и
    посмотрим: используются либо формы $b_i$, либо непосредственно входные
    вершины $x_j$. В силу того, что глубина графа не более $d/t$, различных
    входных вершин не более $2^{d/t}$. Запишем это:
    $$
    y_i = \sum_{k=1}^{m'} \beta_{i,k} b_k(x) + \sum_{j\in\Lambda_i}c_{i,j}x_j,\quad
    |\Lambda_i|\le 2^{d/t}.
    $$
\end{frame}

\begin{frame}
    $$
    y_i = \sum_{k=1}^{m'} \beta_{i,k} b_k(x) + \sum_{j\in\Lambda_i}c_{i,j}x_j,\quad
    |\Lambda_i|\le 2^{d/t}.
    $$
    В матричных терминах:
    $$
    y = Ax,\quad A = \beta B + C,
    $$
    \pause
    где 
    \begin{itemize}
        \item матрица $\beta=(\beta_{i,k})$ размера $n\times m'$,
        \item матрица $B$~--- в которой по строкам стоят коэфф-ты линейных форм
            $b_k(x)$~--- размера $m'\times n$;
        \item в матрице $C$ не более $2^{d/t}$ ненулевых элементов в каждой
            строке;
    \end{itemize}
    \pause

    Значит, мы представили $A$ в виде $\beta B+C$, где $\rank(\beta
    B)\le m'\le m\le sr/\log_2d=s\log_2t/\log_2d$, $\|C\|_0\le n2^{d/t}$,
    $$
    \Rig(A,\frac{s\log_2t}{\log_2d})\le n2^{d/t}.
    $$
\end{frame}

\begin{frame}
    Итак, мы доказали, что если $A$ вычисляется простой схемой, то $A$ не
    слишком жёсткая, т.е. можно приблизить в метрике Хэмминга матрицей малого
    ранга. Заметим, что это приближение \textit{регулярно}, то есть число
    отличий в каждой строке небольшое.
\end{frame}

\begin{frame}{Оценки снизу для конкретных матриц}
    \begin{lemma}
        Пусть $r\ge \log^2n$. Если в матрице $n\times n$ поменять не
        более
        $$
        \frac{n(n-r)}{2r+2}\log\frac{n}{r}
        $$
        эл-тов, то некоторый минор $(r+1)\times (r+1)$ будет без изменений.
    \end{lemma}
    \pause

    Предположим, мы сделали изменения в матрице. Рассмотрим двудольный граф с
    долями $\{v_1,\ldots,v_n\}$ и $\{w_1,\ldots,w_n\}$, где ребро $v_i\mapsto
    w_j$ проводится для тех $(i,j)$, для которых значение $A_{i,j}$ не
    изменилось. При этом количество рёбер в графе не меньше
    $$
    n^2 - \frac{n(n-r)}{2r+2}\log\frac{n}{r}.
    $$
    Нам нужно доказать, что полученный граф содержит $K_{r+1,r+1}$.
    Для этого воспользуемся утверждением из теории графов.
\end{frame}

\begin{frame}
    \begin{lemma}[Zarankiewich problem]
        Если двудольный граф с долями размера $m$ и $n$ не содержит $K_{s,t}$,
        то количество рёбер в нём не превосходит
        $$
        (s-1)^{1/t}(n-t+1)m^{1-1/t}+(t-1)m.
        $$
    \end{lemma}
    \pause
    Матричная формулировка: если в матрице из $\{0,1\}^{m\times n}$ более
    указанного числа единиц, то найдётся подматрица $s\times t$ из одних единиц.
    \pause

    Доказательство.
        Пусть $|V_1|=m$, $|V_2|=n$~--- доли графа.
        Рассмотрим $t$-множества $T\subset V_2$, $|T|=t$. Скажем, что $x\in V_1$
        \textit{покрывает} $T$, если $x$ соединён со всеми элементами $T$.
        \pause

        Каждый $x\in V_1$ покрывает $\binom{d(x)}{t}$ множеств $T$. С другой
        стороны, каждое $T$ покрыто не более чем $(s-1)$ точкой (иначе
        образуется $K_{s,t}$). Следовательно,
        $$
        \sum_{x\in V_1}\binom{d(x)}{t} \le (s-1)\binom{n}{t}.
        $$

\end{frame}

    \begin{frame}
        $$
        \sum_{x\in V_1}\binom{d(x)}{t} \le (s-1)\binom{n}{t}.
        $$

        Посмотрим на биномиальный коэффициент как на многочлен
        $f(u)=\binom{u}{t}=u(u-1)\cdots(u-t+1)/t!$, это выпуклая функция при
        $u\ge t$, следовательно,
        $$
        \binom{m^{-1}\sum d(x)}{t} \le \frac{s-1}{m}\binom{n}{t}.
        $$
        \pause
        Обозначим $y=m^{-1}\sum d(x)=|E|/m$. Тогда
        $$
        \binom{y}{t}\le\frac{s-1}{m}\binom{n}{t}.$$
        \pause
        Ясно, что $y\le n$, поэтому тем более
        $$
        (y-t+1)^t \le \frac{s-1}{m}(n-t+1)^t.
        $$
        Это и есть нужное нам неравенство.


\end{frame}


\begin{frame}
    Следствие: если все миноры матрицы $A$ невырождены, то
    $$
    \Rig(A,r)\ge \frac{n^2}{4r+4}\log\frac{n}{r}
    $$
    при $\log^2 n\le r\le n/2$.


    Примеры:
    \begin{itemize}
        \item матрица Коши $(\frac{1}{x_i+y_j})_{i,j=1}^n$;
        \item $F=(\omega^{ij})$, $\omega$~-- примитивный $n$-корень из единицы.
    \end{itemize}
\end{frame}


\begin{frame}{Матрицы Уолша--Адамара}
    Вспомним про матрицы Уолша--Адамара $H^n$:
    \pause
    \begin{itemize}
        \item размера $2^n\times 2^n$ с элементами $\pm 1$;
        \item $H^n(x,y)=(-1)^{\langle x,y\rangle}$, $x,y\in\{0,1\}^n$;
        \item строки и столбцы ортогональны;
        \item сложная для коммуникации даже с неограниченной ошибкой: $U(H^n)\ge
            cn$.
    \end{itemize}
\end{frame}

\begin{frame}

    Докажем следующую оценку жесткости:
    $$
    \Rig(H^n,r)\ge \frac{N^2}{4r},\quad\mbox{где $N=2^n$},
    $$
    при условии что $r$ это степень двойки.
    \pause

    Зафиксируем $x_1$ и $y_1$, получим разбиение $H^n$ на
    четыре подматрицы $\pm H^{n-1}$.
    \pause
    
    Обобщим: возьмём $s\ge 1$ и разделим $H^n$ на подматрицы $\pm H^s$.
    Получится $N^2/2^{2s}$ штук. Если мы делаем меньше $N^2/4r$ изменений, то в
    одной из подмариц изменится менее $2^{2s}/4r$ элементов.
    \pause

    Полагаем $2^s=2r$, тогда на одну из матриц $\pm H^s$, которая имеет ранг
    $2^s=2r$, приходится менее $2^{2s}/4r=r$ изменений и у неё останется
    $\rank\ge r$.
    \pause\vspace{5pt}

    Такая оценка впервые была получена в работе Б.С.Кашина и А.А.Разборова
    (1998) для общих матриц Адамара. Отметим, что она недостаточна для
    Valiant-жёсткости.

\end{frame}

\begin{frame}
    В работе 2016 года J.Alman, R. Williams получили неожиданный результат --
    матрицы Уолша--Адамара не являются жёсткими!
    \pause

    \begin{theorem}[Alman, Williams]
        Для любого поля $\mathbb F$, достаточно малого $\eps>0$, 
        $N=2^n$,
        имеем
        $$
        \Rig^{\mathbb{F}}(H^n,N^{1-c\eps^2})\le N^{1+c\eps\log(1/\eps)}.
        $$
    \end{theorem}

\end{frame}

\begin{frame}{Доказательство (случай $\mathbb F\subset\mathbb Q$)}

    Пусть $p(x,y)$~--- полином от $2n$ переменных ($x,y\in\{0,1\}^n$), состоящий из
    $m$ мономов. Тогда матрица
$$
    M(x,y)=p(x,y),\quad x,y\in\{0,1\}^n,
$$
    имеет $\rank M\le m$.\pause

    Действительно, каждый моном имеет вид $u(x)v(y)$ и представляет одноранговую
    матрицу.
\pause\vspace{5pt}

    В нашем случае можно взять $p(x,y)=R(x_1y_1+\ldots+x_ny_n)$,
    где полином $R$ альтернирует: $R(j)=(-1)^j$ для $2n\eps \le j \le
    (1+\eps)n$.

    \pause
    Чтобы уменьшить количество мономов, мы заменим $x_i^my_i^m\mapsto x_iy_i$;
    это не изменит значения полинома для булевых векторов.
    \pause

    Количество мономов оценивается
    $$
        m(p)\le \sum_{s=0}^{\deg Q}\binom{n}{s} \le
        2^{h_2(r/n)}\le 2^{n(1-c\eps^2)}.
    $$
    Полагаем $M(x,y)=p(x,y)$, $\rank M\le m(p)$.
\end{frame}

\begin{frame}
    По построению $M(x,y)=H^n(x,y)$
    при $\langle x,y\rangle \in [2n\eps,(1/2+\eps)n]$:
    $$
    M(x,y)=p(x,y)=Q(\langle x,y\rangle)=(-1)^{\langle x,y\rangle}.
    $$
    \pause

    Исправление $M$: полагаем $M'(x,y)=M(x,y)$ для ``ядра''
    $\|x\|_1,\|y\|_1\in[(1/2-\eps)n,(1/2+\eps)n]$, и $M'(x,y)=H^n(x,y)$ иначе.
    \pause

    Изменения касаются малого кол-ва строк/столбцов и не сильно увеличат ранг.
    \pause

    Отличие $M'(x,y)\ne H^n(x,y)$ может быть только для пар $(x,y)$ вне ядра и
    только для $\langle x,y\rangle\not\in[2n\eps,(1/2+\eps)n]$. Следовательно,
    все расхождения содержат среди пар $(x,y)$:
    $$
    \begin{cases}
        \langle x,y\rangle < 2n\eps,\\
        \|x\|_1 \in [(1/2-\eps)n,(1/2+\eps)n],\\
        \|x\|_1 \in [(1/2-\eps)n,(1/2+\eps)n],\\
    \end{cases}
    \eqno{(*)}
    $$
    Упражнение: для фиксированного $x$ существует не более $2^{c\eps\log(1/\eps)n}$ таких
    $y$, что выполнено (*). Матрица $M'$ даёт нужное приближение для $H^n$.

\end{frame}


\begin{frame}
    Примёр жёсткой матрицы. Пусть $p_{i,j}$, $1\le i,j\le n$~--- различные
    простые числа (например, первые $n^2$ простых).
    Рассмотрим матрицу $P=(\sqrt{p_{i,j}}) \in \R^{n\times n}$.
    \pause
    \begin{statement}
        $\Rig(P,n/17)\ge n^2/17$.
    \end{statement}
    \pause

    Это утверждение следует из теоремы
    \begin{theorem}
        Пусть $A\in\R^{n\times n}$ и $1\le r\le n$. Если любые $nr$ произведений
        различных элементов $A$ линейно независимы над $\mathbb Q$, то
        $$
        \Rig(A,r)\ge n(n-16r).
        $$
    \end{theorem}
    \pause

    Известно, что все числа вида $\sqrt{k}$, где $k$ свободно от квадратов,
    линейно независимы над $\mathbb Q$. Следовательно, к матрице $P$ применима
    данная теорема. 
\end{frame}


\begin{frame}
    Для доказательства нам потребуется несколько определений.
    \pause
    Пусть $X=(a_1,\ldots,a_p)$~--- последовательность чисел и $t\in\mathbb
    N$. Определим размерности Shoup--Smolensky:
    $$
    D_t(X) := \dim_{\mathbb Q}\langle a_{i_1}\cdots a_{i_t}\colon 1\le
    i_1<i_2<\ldots<i_t\le p\rangle,
    $$
    \pause
    $$
    D_t^*(X) := \dim_{\mathbb Q}\langle a_{i_1}\cdots a_{i_t}\colon 1\le
    i_1\le i_2\le \ldots\le i_t\le p\rangle.
    $$
    \pause
    Выполнены простые свойства:
    \begin{itemize}
        \item $D_t(X)\le D_t^*(X)$;
            \pause
        \item $D_t(X)\le\binom{p}{t}$;
            \pause
        \item $D_t^*(X)\le\binom{p+t-1}{t}$.
            \pause
        \item Пусть $X=(a_1,\ldots,a_p)$, $Y=(b_1,\ldots,b_q)$,
            $XY:=(a_ib_j)_{\substack{1\le i\le p\\1\le j\le q}}$. Тогда
            $$
            D_t^*(XY) \le D_t^*(X)D_t^*(Y).
            $$
            \pause
            \textcolor{red}{Верно ли это для $D_t$?}
    \end{itemize}
\end{frame}

\begin{frame}
    Для матрицы $A\in\R^{m\times n}$ величины $D_t(A)$ и $D_t^*(A)$ определяются
    как соответствующие размерности для списка элементов матрицы (в произвольном
    порядке).
    \pause
    Если произведение матриц $AB$ определено, то
    $$
    D_t^*(AB)\le D_t^*(A)D_t^*(B).
    $$
    \pause
    \begin{statement}
        Если $A\in\R^{m\times n}$ и $\rank A=r$, то
        $$
        D_t^*(A) \le \binom{mr+t-1}{t}\binom{nr+t-1}{t}.
        $$
    \end{statement}
    \pause
    Действительно, матрица ранга $r$ представляется в виде $A=BC$ размеров
    $m\times r$ и $r\times n$. Далее применяем оценку 
    $D_t^*(X)\le\binom{|X|+t-1}{t}$.
\end{frame}


\begin{frame}
    Вернёмся к доказательству теоремы. Нам дано, что все произведения $nr$
    элементов линейно независимы. Нужно оценить жёсткость.
    \pause

    Предположим, $A=B+C$, где $\rank B\le r$ и $\|C\|_0\le R$; оценим $R$ снизу.
    \pause

    $$
    D_t(B)\le D_t^*(B) \le \binom{nr+t}{t}^2.
    $$
    \pause
    С другой стороны, если рассматривать произведения элементов $B$, где
    $C_{i,j}=0$ (и, следовательно, $B_{i,j}=A_{i,j}$, то они также линейно
    независимы над $\mathbb Q$. Следовательно,
    $$
    D_t(B) \ge \binom{n^2-R}{t}.
    $$
    \pause
    Сравним неравенства:
    $$
    \binom{n^2-R}{t} \le \binom{nr+t}{t}^2.
    $$
\end{frame}

\begin{frame}
    Положим $t=nr$:
    $$
    \binom{n^2-R}{nr} \le \binom{2nr}{nr}^2 \le 2^{4nr}.
    $$
    \pause
    Воспользуемся полезным неравенством
    $$
    (n/k)^k \le \binom{n}{k} \le (en/k)^k,\quad 1\le k\le n.
    $$
    Получим
    $$
    ((n^2-R)/nr)^{nr} \le 2^{4nr},
    $$
    $$
    (n^2-R)/nr \le 16,\quad R \ge n^2-16nr,
    $$
    Ч.т.д.
\end{frame}


\begin{frame}

    \begin{thebibliography}{XXXX}
        \bibitem{L09} S.V. Lokam, ``Complexity Lower Bounds using Linear Algebra'',
            2009.
        \bibitem{V77} L.~Valiant, ``Graph-theoretic arguments in low-level
            Complexity'', 1977.
        \bibitem{AW16} J.~Alman, R.~Williams, ``Probabilistic Rank and Matrix
                Rigidity'', 2016, arXiv:1611.05558.
    \end{thebibliography}


\end{frame}

\end{document}
